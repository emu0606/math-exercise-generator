\documentclass[a4paper,11pt]{article}
\usepackage{xeCJK}
\usepackage[margin=1.8cm]{geometry}
\usepackage{tikz}
\usetikzlibrary{positioning,shapes,arrows,decorations.pathmorphing,calc,shadows,decorations.markings}
\usepackage{amsmath,amssymb}
\usepackage{multicol}
\usepackage{enumitem}
\usepackage{fancyhdr}
\usepackage{xcolor}
\usepackage{mdframed}
\usepackage[most]{tcolorbox}
\usepackage{fontspec}

% 設定字體
\setCJKmainfont{Noto Sans TC}
\setmainfont{Noto Sans TC}
\setmonofont{Noto Sans TC}

% 全局字體設置
\renewcommand{\normalsize}{\fontsize{11pt}{15pt}\selectfont}
\renewcommand{\large}{\fontsize{14pt}{18pt}\selectfont}
\renewcommand{\Large}{\fontsize{16pt}{20pt}\selectfont}
\renewcommand{\huge}{\fontsize{20pt}{24pt}\selectfont}

% 基本設定
\pagestyle{fancy}
\fancyhf{}
\renewcommand{\headrulewidth}{0pt}
\fancyfoot[C]{\thepage}
\setlength{\parskip}{0.5em}
\setlength{\parindent}{0em}
\setlist[enumerate]{leftmargin=*,labelsep=0.5em,topsep=0.3em,itemsep=0.2em}

% 定義顏色
\definecolor{accent1}{RGB}{41,128,185}
\definecolor{accent2}{RGB}{231,76,60}
\definecolor{accent3}{RGB}{46,204,113}
\definecolor{accent4}{RGB}{155,89,182}
\definecolor{accent5}{RGB}{243,156,18}
\definecolor{accent6}{RGB}{52,73,94}

% 樣式1 - 藍色圓角矩形
\newcommand{\qnumOne}[1]{%
  \tikz[remember picture]{
    \node[rectangle, rounded corners=3pt, fill=accent1, text=white, inner sep=2pt] (qnum) {\bfseries #1};
  }
}

% 樣式7 - 帶陰影的圓角矩形
\newcommand{\qnumSeven}[1]{%
  \tikz[remember picture]{
    \node[rectangle, rounded corners=3pt, fill=white, draw=accent6, drop shadow, inner sep=2pt] (qnum) {\bfseries #1};
  }
}

% 樣式17 - 圓角標籤
\newcommand{\qnumSeventeen}[1]{%
  \tikz[remember picture]{
    \node[rectangle, rounded corners=5pt, fill=accent3!10, draw=accent3, inner sep=2pt] (qnum) {\bfseries #1};
  }
}

% 樣式18 - 雙色標籤
\newcommand{\qnumEighteen}[1]{%
  \tikz[remember picture]{
    \node[rectangle, left color=accent4!20, right color=accent5!20, 
          draw=accent4, rounded corners=2pt, inner sep=2pt] (qnum) {\bfseries #1};
  }
}

% 題目框與題號分開,並讓題號與題目框左上角重合
\newcommand{\questionWithLabel}[2]{%
  \begin{tikzpicture}
    % 繪製題目框(沒有底色)
    \draw[rounded corners=3pt, thick] (0,0) rectangle (4,-2.47);
    
    % 放置題號在左上角
    \node[anchor=north west, inner sep=0] at (0,0) {#1};
    
    % 放置題目內容(上距增加,不與題號在同一行)
    \node[anchor=north west, text width=3.8cm, inner sep=0.2cm] at (0,-0.4) {#2};
  \end{tikzpicture}
}

\begin{document}
\noindent 題目樣式比較 \hfill 2025年4月26日 \hfill 姓名:\_\_\_\_\_\_\_\_\_

\begin{center}
\Large\textbf{不同題號樣式比較(黃金矩形排版)}
\end{center}

\vspace{0.5cm}

\begin{center}
\begin{tabular}{p{4.2cm}p{4.2cm}p{4.2cm}}
% 第一排:樣式1(藍色圓角矩形)
\questionWithLabel{\qnumOne{1}}{計算:$3^2 + 4 \times 5 - 8 \div 2$} &
\questionWithLabel{\qnumOne{2}}{一個三角形的三個角度比是1:2:3,求最大角的度數。} &
\questionWithLabel{\qnumOne{3}}{寫出三個英文形容詞來描述春天。} \\[0.7cm]

% 第二排:樣式7(帶陰影的圓角矩形)
\questionWithLabel{\qnumSeven{4}}{一個長方形面積為24平方公尺,長是寬的兩倍,求周長。} &
\questionWithLabel{\qnumSeven{5}}{請寫出二氧化碳的化學式。} &
\questionWithLabel{\qnumSeven{6}}{翻譯:\textit{The book is on the table.}} \\[0.7cm]

% 第三排:樣式17(圓角標籤)
\questionWithLabel{\qnumSeventeen{7}}{地球與太陽的平均距離約為多少?} &
\questionWithLabel{\qnumSeventeen{8}}{台灣的國花是什麼?} &
\questionWithLabel{\qnumSeventeen{9}}{列出兩種可再生能源。} \\[0.7cm]

% 第四排:樣式18(雙色標籤)
\questionWithLabel{\qnumEighteen{10}}{水的化學式是什麼?} &
\questionWithLabel{\qnumEighteen{11}}{請舉出兩個對環境友善的日常行為。} &
\questionWithLabel{\qnumEighteen{12}}{計算:$\frac{1}{4} + \frac{1}{8} = ?$}
\end{tabular}
\end{center}

\vfill
\begin{center}
\small{比較注意事項:觀察每種題號樣式的清晰度、美觀性以及與題目文字的協調性}
\end{center}

\end{document}